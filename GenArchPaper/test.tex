\documentclass[]{article}
\usepackage{lmodern}
\usepackage{amssymb,amsmath}
\usepackage{ifxetex,ifluatex}
\usepackage{fixltx2e} % provides \textsubscript
\ifnum 0\ifxetex 1\fi\ifluatex 1\fi=0 % if pdftex
  \usepackage[T1]{fontenc}
  \usepackage[utf8]{inputenc}
\else % if luatex or xelatex
  \ifxetex
    \usepackage{mathspec}
    \usepackage{xltxtra,xunicode}
  \else
    \usepackage{fontspec}
  \fi
  \defaultfontfeatures{Mapping=tex-text,Scale=MatchLowercase}
  \newcommand{\euro}{€}
\fi
% use upquote if available, for straight quotes in verbatim environments
\IfFileExists{upquote.sty}{\usepackage{upquote}}{}
% use microtype if available
\IfFileExists{microtype.sty}{%
\usepackage{microtype}
\UseMicrotypeSet[protrusion]{basicmath} % disable protrusion for tt fonts
}{}
\usepackage[margin=1in]{geometry}
\ifxetex
  \usepackage[setpagesize=false, % page size defined by xetex
              unicode=false, % unicode breaks when used with xetex
              xetex]{hyperref}
\else
  \usepackage[unicode=true]{hyperref}
\fi
\hypersetup{breaklinks=true,
            bookmarks=true,
            pdfauthor={Heather E. Wheeler1,2, GTEx Consortium, Kaanan P. Shah3, . . ., Nancy J. Cox4, Dan L. Nicolae3, Hae Kyung Im3},
            pdftitle={An atlas of the genetic architecture of gene expression traits across the entire human body},
            colorlinks=true,
            citecolor=blue,
            urlcolor=blue,
            linkcolor=magenta,
            pdfborder={0 0 0}}
\urlstyle{same}  % don't use monospace font for urls
\setlength{\parindent}{0pt}
\setlength{\parskip}{6pt plus 2pt minus 1pt}
\setlength{\emergencystretch}{3em}  % prevent overfull lines
\setcounter{secnumdepth}{0}

%%% Use protect on footnotes to avoid problems with footnotes in titles
\let\rmarkdownfootnote\footnote%
\def\footnote{\protect\rmarkdownfootnote}

%%% Change title format to be more compact
\usepackage{titling}

% Create subtitle command for use in maketitle
\newcommand{\subtitle}[1]{
  \posttitle{
    \begin{center}\large#1\end{center}
    }
}

\setlength{\droptitle}{-2em}
  \title{An atlas of the genetic architecture of gene expression traits across
the entire human body}
  \pretitle{\vspace{\droptitle}\centering\huge}
  \posttitle{\par}
  \author{Heather E. Wheeler\textsuperscript{1,2}, GTEx Consortium, Kaanan P.
Shah\textsuperscript{3}, . . ., Nancy J. Cox\textsuperscript{4}, Dan L.
Nicolae\textsuperscript{3}, Hae Kyung Im\textsuperscript{3}}
  \preauthor{\centering\large\emph}
  \postauthor{\par}
  \predate{\centering\large\emph}
  \postdate{\par}
  \date{\textsuperscript{1}Department of Biology and
\textsuperscript{2}Department of Computer Science, Loyola University
Chicago, \textsuperscript{3}Section of Genetic Medicine, Department of
Medicine, University of Chicago, \textsuperscript{4}Division of Genetic
Medicine, Vanderbilt University 2015-11-23 15:36:22}

\usepackage{setspace}
\setstretch{2}
\usepackage{soul}


\begin{document}

\maketitle


Using the hybrid polygenic-sparse approach of BSLMM (Bayesian Sparse
Linear Mixed Model) {[}1{]}, we show that the local architecture of gene
expression is sparse (high PGE) for most heritable genes in both DGN and
GTEx. Using the elastic net {[}2{]}, we observed improved
cross-validated expression prediction for \(\alpha \geq 0.5\) across
tissues, confirming the sparsity result. This result demonstrates that
sparse effects can be identified with sample sizes in the hundreds
rather than the thousands and is supported by many prior studies with
sample sizes near 100 that identified replicable eQTLs near the
transcription start sites of genes{[}1{]}. Conversely, for traits that
are highly polygenic, e.g.~height, BMI, schizophrenia, and bipolar
disorder, thousands to tens of thousands of samples are needed to
identify significant genetic signals {[}\textbf{???}{]}. Therefore, the
distal contributions to expression h\textsuperscript{2} are likely to be
more polygenic because they could not be accurately estimated here with
sample sizes in the hundreds.

1. Zhou X, Carbonetto P, Stephens M. Polygenic modeling with bayesian
sparse linear mixed models. Visscher PM, editor. PLoS Genetics. Public
Library of Science (PLoS); 2013;9: e1003264.
doi:\href{http://dx.doi.org/10.1371/journal.pgen.1003264}{10.1371/journal.pgen.1003264}

2. Zou H, Hastie T. Regularization and variable selection via the
elastic net. Journal of the Royal Statistical Society: Series B
(Statistical Methodology). Wiley-Blackwell; 2005;67: 301--320.
doi:\href{http://dx.doi.org/10.1111/j.1467-9868.2005.00503.x}{10.1111/j.1467-9868.2005.00503.x}

\end{document}
